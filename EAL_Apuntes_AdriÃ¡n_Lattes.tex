\documentclass[10pt, spanish]{report}
\usepackage[none]{hyphenat}
\usepackage[utf8]{inputenc}
\usepackage[T1]{fontenc}
\usepackage[spanish]{babel}

\usepackage{geometry}
\def \margin {20mm}
\geometry{
    a4paper,
    left=\margin,
    right=\margin,
    top=\margin,
    bottom=\margin
}

\usepackage{lmodern}
\usepackage{xfrac}

\usepackage{enumitem}
\renewcommand{\labelenumi}{(\theenumi)}
\renewcommand{\labelenumii}{(\theenumii)}

\usepackage{multicol}

% \setitemize{label=---,leftmargin=0.6cm}

\usepackage{amssymb}
\usepackage{mathtools}

\usepackage{marginnote}

%For left superscripts
\usepackage{leftidx}

\makeatletter
\renewcommand{\@makechapterhead}[1]{%
\vspace*{50 pt}%
{\setlength{\parindent}{0pt} \raggedright \normalfont
\bfseries\huge\thechapter.\ #1
\par\nobreak\vspace{1em}}}
\makeatother

% Theorems
\usepackage{amsthm}
\usepackage{thmtools}
\newtheorem*{lema}{Lema}
\newtheorem*{prop}{Proposición}
\newtheorem*{tma}{Teorema}
\newtheorem*{cor}{Corolario}
\theoremstyle{definition}
\newtheorem*{defin}{Definición}
\newtheorem*{notacion}{Notación}
% \newtheoremstyle{custom}{\topsep}{\topsep}{}{}{}{.}{ }{}
% \theoremstyle{custom}
\newtheorem*{ej}{Ejemplo}
\newtheorem*{ejer}{Ejercicio}
\newenvironment{sol}{\textit{Solución.}}{\hfill$\square$}
% \theoremstyle{remark}
\newtheorem*{obs}{Observación}
\newtheorem*{nota}{Nota}

% No indentation: \setlength\parindent{0pt}
\usepackage{parskip}

\let\emptyset\varnothing

% Custom math commands
\newcommand{\N}{\mathbb{N}}
\newcommand{\Z}{\mathbb{Z}}
\newcommand{\Q}{\mathbb{Q}}
\newcommand{\R}{\mathbb{R}}
\newcommand{\C}{\mathbb{C}}
\newcommand{\F}{\mathbb{F}}
\newcommand{\K}{\mathbb{K}}
\let\L\undefined
\newcommand{\L}{\mathbb{L}}

\newcommand{\id}{\text{id}}
\newcommand{\im}[1]{\operatorname{im}\left(#1\right)}

\renewcommand{\geq}{\geqslant}
\renewcommand{\leq}{\leqslant}

\DeclareMathSymbol{*}{\mathbin}{symbols}{"01}

\newcommand{\fecha}[1]{\marginpar{\underline{#1}}}
% \newcommand{\fecha}[1]{{\hfill\large{#1/2021}}\\[-0.8em]\noindent\rule{17cm}{0.1mm}}
\newcommand{\completar}{\fbox{\textbf{¡Completar!}}}
\newcommand{\revisar}{\fbox{¡Revisar!}}

\title{Ecuaciones algebraicas}
\author{Apuntes de Adrián Lattes}
\date{2021}

\begin{document}
\maketitle

\chapter{Extensiones de cuerpos}
\fecha{15/02}
\section{Raíces de polinomios y extensiones de cuerpos}

\begin{defin}
    Decimos que un anillo unitario y conmutativo $(\F,+,*)$ es un
    \textit{cuerpo} si para todo $a\in \F\setminus\{0\}$ existe $a^{-1}\in
    \F\setminus\{0\}$ tal que $a*a^{-1}=1$ y $1\neq 0$.
\end{defin}

\begin{obs}
    Con la definición anterior, el anillo $\{0\}$ no es un cuerpo.
\end{obs}

\begin{prop}
    Si $\F$ es un cuerpo sus únicos ideales son $\left\{ 0 \right\}$ y $\F$.
\end{prop}
\begin{proof}
    Sea $I\subseteq \F$ un ideal distinto de $\left\{ 0 \right\}$ y $a\in
    I\setminus\{0\}$. Entonces todo $b\in\F$ se puede escribir como
    $b=(ba^{-1})a\in I$, de donde $\F\subseteq I$ y por tanto $I=\F$.
\end{proof}

\begin{defin}
    Un homomorfismo de anillos unitarios $f:A\to A'$ es una aplicación tal que
    para todos $a,b\in A$:
    \begin{enumerate}
        \item $f(a+a')=f(a)+f(a')$
        \item $f(a*a')=f(a)*f(a')$
        \item $f(1_A)=f(1_{A'})$
    \end{enumerate}
\end{defin}

\begin{lema}
    Si $\F$ y $\F'$ son dos cuerpos y $f: \F\to\F'$ es un homomorfismo de
    anillos (unitarios), entonces $f$ es inyectivo.
\end{lema}
\begin{proof}
    Sabemos por definición que $f(1)=1$, por lo que $1_\F\neq \ker{f}$ y
    $\ker{f}\neq \F$. Como además $\ker{f}$ es un ideal de $\F$ se tiene que
    $\ker{f}=\{0\}$ lo que equivale a que $f$ sea inyectivo.
\end{proof}

\begin{defin}
    Sean $\K$ y $\L$ cuerpos y $f:\K\to\L$ un homomorfismo de anillos. Decimos
    que $f$ define una \textit{extensión de $\K$ en $\L$}.
    Por el lema tenemos que $\K\cong f(\K)\subseteq \L$, por lo que identificaremos
    $\K$ con $f(\K)$ y escribiremos \[\K\subseteq \L.\]
\end{defin}

\begin{ej}
    $\Q\subseteq\R$, $\R\subseteq\C$.
\end{ej}

\fecha{16/02}

\begin{ej}
    Sea $\F$ un cuerpo, consideramos el anillo de polinomios $\F[x]$ y
    $\K$ el cuerpo de fracciones de $\F[x]$,
    \[\K=\left\{ \frac{p}{q}\mid p,q\in \F[x], q\neq 0 \right\}\]
    donde $\frac{p}{q} = \frac{p'}{q'}$ si $pq'=p'q$.
    En $\K$ definimos la suma y producto como
    \begin{align*}
        \frac{p}{q}+\frac{p'}{q'}&=\frac{pq'+p'q}{pq}\\
        \frac{p}{q}*\frac{p'}{q'}&=\frac{pp'}{qq'}
    \end{align*}
\end{ej}

\begin{obs}
    $\frac{p}{q}$ es la clase de equivalencia de $(p,q)$ via la relación
    $(p,q)\sim(p',q')$ donde $(p,q),(p',q')\in K[x]\times(K[x]\setminus\{0\})$
\end{obs}

\begin{notacion}
    Sea $\K$ un cuerpo denotamos el cuerpo de fracciones de $\K[x]$ como $\K(x)$.
\end{notacion}

\begin{ej}
    Si $\F=\Q$, $\frac{2+x+x^3}{1+x^2}\in \Q(x)$
\end{ej}

\begin{ejer}
    Sean $\K$ y $\F$ dos cuerpos. La aplicación $j:\F\to\K$ tal que $j(a)=
    \frac{a}{1}$ es un homomorfismo que define la extensión $\F\subseteq \K$.
\end{ejer}


\section{Construcción de $\C$ a partir de $\R$}

Sea $\C =\left\{ a+bi\mid a,b\in\R, i^2 = -1 \right\}$ veamos que $\C\cong
\frac{\R[x]}{I}$ donde $I=\langle x^2 +1\rangle=\left\{(x^2+1)g\mid g\in\R[x]
\right\}$. La relación de congruencia está definida para $p,q\in\R[x]$ como
\[p\sim_I q \Leftrightarrow p-q\in I\Leftrightarrow x^2+1\mid(p-q).\]

Dado $p\in\R[x]$ se puede escribir como $p=h(x)*(x^2+1)+bx+a$, de donde
$p\sim_I bx+a=r(x)$ y su clase de equivalencia es $p+I=\left\{ p\mid p-q\in
I\right\}=(a+bx)I$. Con esto vemos que el conjunto cociente
\[\frac{\R[x]}{I}=\left\{ p+I\mid p\in\R[x] \right\} \]
es un anillo con las operaciones
\begin{align*}
    (p+I)+(p'+I)&=(p+p')+I\\
    (p+I)*(p'+I)&=(p*p')+I
\end{align*}

\begin{obs}
    La aplicación $j:\R\to\frac{\R[x]}{I}$ tal que $a\mapsto a+I$ es un
    homomorfismo inyectivo de anillos.
\end{obs}

En lo siguiente denotamos la clase $a+I$ por $a$, para todo $a\in\R$ y
escribimos $\alpha = x+I$. Con esto la clase de un elemento $a+bx$ es
\[a+bx+I=a+I+(b+I)*(\alpha+I)=a+b\alpha\]
y las operaciones
\begin{align*}
    (a+b\alpha)+(a'+b'\alpha)&=a+a'+(b+b')\alpha\\
    (a+b\alpha)*(a'+b'\alpha)&=aa'+(ab'+a'b)\alpha+bb'\alpha^2
\end{align*}
Ahora $\alpha^2=(x+I)^2=x^2+I$ y $x^2=1(x^2+1)=-1$ por lo que $\alpha^2=x^2+I
=-1+I=-1$

Con esto tenemos que la aplicación $\frac{\R[x]}{I}\to \C$ tal que
$a+b\alpha\mapsto a+bi$ es un isomorfismo.

\section{El anillo cociente $\K[x]/\left<f\right>$}

Sea ahora $\K$ un cuerpo, $f\in\K[x]$ de grado $n\geq 1$ e $I=\langle f\rangle=
\left\{ g*f\mid g\in\K[x] \right\}$, consideramos la congruencia módulo $I$:
\[p\sim_I q\Leftrightarrow p-q\in I \Leftrightarrow f\mid(p-q)\]
donde $p,q\in\K[x]$. La clase de equivalencia de $p\in\K[x]$ es
$p+I\coloneqq\left\{q\in\K[x]\mid p\sim_Iq\right\}=\left\{p+g\mid g\in I\right\}$.

Se tiene entonces que \[\frac{\K[x]}{I}=\left\{ p+I\mid p\in\K[x] \right\}\] es un
anillo unitario y conmutativo con las operaciones definidas entre los
representantes. Así
\[\frac{\K[x]}{I}=\left\{ a_0+a_1x+\ldots+a_{n-1}x^{n-1}+I\mid a_0,\ldots,a_{n-1}
\in\K \right\}\]
donde las $a_i$ son las clases de los restos de dividir por $f$ y $n=\deg{f}$.
Si denotamos $a+I$ por $a$ para todo $a\in\K$ y $x+I$ por $\alpha$ podemos
escribir
\[\frac{\K[x]}{I}=\left\{ a_0+a_1\alpha+\ldots+a_{n-1}\alpha^{n-1}\mid a_0,\ldots,a_{n-1}
\in\K \right\}.\]


Además $f(\alpha)=0$ en $\frac{\K[x]}{I}$ ya que, si $f=c_0+c_1x+\ldots+c_nx^n$:
\begin{equation*}
    \begin{split}
        f(\alpha) &=c_0+c_1\alpha+\ldots+c_n\alpha^n
              =c_0+I+(c_1+I)(x+I)+\ldots+(c_n+I)(x+I)^n \\
                  &=(c_0+c_1x+\ldots+c_n+x^n)+I=f+I=0+I.
    \end{split}
\end{equation*}

\fecha{17/02}
\begin{defin}
    Un anillo $A$ es \textit{íntegro} si para todos $a,b\in A$ con $a\neq 0, b
    \neq 0$ entonces $ab\neq 0$.
\end{defin}

\newpage
\begin{prop}
    Sea $\K$ un cuerpo y $f\in \K$ de grado $n\geq1$. Son equivalentes
    \begin{enumerate}
        \item $\frac{\K[t]}{\left< f \right>}$ es un anillo íntegro.
        \item $\frac{\K[t]}{\left< f\right>}$ es un cuerpo.
        \item $f$ es irreducible.
    \end{enumerate}
\end{prop}

\begin{proof}\hspace{1em}
    \begin{itemize}[itemindent=36pt]
        \item[(2)$\implies$(1)] Todo cuerpo es anillo íntegro.
        \item[(1)$\implies$(3)] Si $f=g*h$ con $\deg{f}>\deg{g},\deg{f}\geq
            \deg{h}$ y sea $\alpha = x+\left< f \right> $ se tiene $f(\alpha)=0
            =g(\alpha)h(\alpha)$. Ahora $g+\left< f \right> = g(\alpha) \neq 0$
            ya que $f$ divide a $g$ y análogamente $h+\left< f \right> =
            h(\alpha)\neq 0$.
            Por tanto $A=\frac{\K[t]}{\left< f \right> }$ no es íntegro.
        \item[(3)$\implies$(2)] Sea $g\in \K[t]$  tal que $g+\left< f \right>
            =g(\alpha)\neq 0$. Vemos que $\exists(g(\alpha))^{-1}$ en
            $A=\frac{\K[t]}{\left< f \right> }$. Como $f$ es irreducible y
            $\neg f|g \implies \gcd(f,g)=1$. En $\K[t]$ tenemos la identindad de
            Bézout (gracias a que $\K$ es cuerpo), por lo que existen $p,q\in
            \K[t]$ tales que $pf+qg=1$.
            Ahora, como $f(\alpha)=0$, $q(\alpha)=q+\left< f \right>$  es el
            inverso de $g(\alpha)=g+\left< f \right>$ ya que $1=q(\alpha)
            g(\alpha)$.
    \end{itemize}
    \vspace{-1em}
\end{proof}

\begin{cor}
    Sean $\K$ un cuerpo y $f\in\K[t]$ irreducible, existe una extensión
    $\K\subseteq\L$ y $\alpha\in\L$ tales que $f(\alpha)=0$.
\end{cor}

\begin{proof}
    Tomamos el cuerpo $\frac{\K[t]}{\left< f \right> }$ y el polinomio
    $\alpha=t+\left< f \right>$  y aplicamos la proposición anterior.
\end{proof}

\begin{ej}
    Sea $f=t^2-2\in \Q[t]$. Si $\alpha=\sqrt{2}$, $f$  tiene una raíz en $\R$ y
    $f=(t-\sqrt{2})(t+\sqrt{2})$ en $\R[t]$.
\end{ej}

\begin{defin}
    Sea $\K$ un cuerpo y $f\in\K[t]$, se dice que $f$ se \textit{escinde} sobre
    $\K$ (o se \textit{descompone totalmente} sobre $\K$) si existen $\alpha_1,
    \ldots,\alpha_n\in\K$ y $a\in\K$ tales que
    \[f=a(t-\alpha_1)*\ldots*(t-\alpha_n).\]
\end{defin}

\begin{obs}
    La condición de la definición es equivalente a que $f$ tenga todas sus
    raíces en $\K$. Es decir, todos los factores irreducibles de $f$ en $\K[t]$
    tienen grado $1$.
\end{obs}

\begin{tma}
    Sea $\K$ un cuerpo y $f\in\K[t]$ de grado $n\geq1$. Entonces existe una
    extensión $\K\subseteq L$ tal que $f$ se escinde sobre $L$.
\end{tma}

\begin{proof}
    Por inducción sobre $n=\deg{f}$:
    \begin{itemize}[itemindent=30pt]
        \item[Si $n=1$] $f=a+bt$ con $b\neq 0$ y $f=b(\frac{a}{b}+t)$ y
            $-\frac{a}{b}$ es raíz de $f$.
        \item[Si $n>1$] Supongamos que el resultado es cierto para polinomios de
            grado $n-1$. Como $\K[t]$ es D.F.U. puedo expresar $f$
            como producto de irreducibles: $f=f_1*\ldots*f_s$ con $f_i\in\K[t]$
            irreducibles. Por el corolario anterior existe una extensión
            $\K\subseteq\L_1$ tal que existe $\alpha_1\in\L_1$ tal que
            $f_1(\alpha_1)=0$ lo que equivale a que $f_1=(t-\alpha_1)*g$ en
            $\L_1[t]$.

            Ahora $f=(t-\alpha_1)*g*f_2*\ldots*f_s$ en $\L_1[t]$ donde
            $h=g*f_2*\ldots f_s\in\L_1[t]$ tiene $\deg{h}=n-1$. Por tanto,
            aplicando la hipótesis de inducción, se tiene que existe una
            extensión $\L_1\subseteq\L$ tal que $h$ se escinde sobre $\L$, de
            donde tenemos que $f$ se escinde sobre $\L$.
    \end{itemize}
    \vspace{-1em}
\end{proof}

\begin{ej}
    Sea $f=t^4-2$ entonces
    \[f=(t-\sqrt[4]{2})(t+\sqrt[4]{2})(t-i\sqrt[4]{2})(t+i\sqrt[4]{2})\]
    lo que implica que $f$ se escinde sobre $\C$.
\end{ej}

\begin{tma}[fundamental del Álgebra]
    Si $f\in \C[t]$ de grado $n$, entonces $f$ se escinde sobre $\C$.
\end{tma}

\begin{defin}
    Un cuerpo $\K$ es \textit{algebraicamente cerrado} si todo polinomio $f\in\K[t]$ de
    grado $\deg{f}\geq1$ se escinde sobre $\K$.
\end{defin}

\begin{ej}
    $\C$ es algebraicamente cerrado.
\end{ej}

\begin{ej}
    Sean $\K=\frac{\Z}{2\Z}=\left\{0,1\right\}$ y $f=t^3+t+1\in\K[t]$. Como
    $\deg{f}=3$ y no tiene factores de grado 1 (no tiene raíces en $\K$) tenemos
    que $f$ es irreducible en $\K[t]$. Además
    \[\L=\frac{\K[t]}{\left< f \right> }=\left\{ a_0+a_1\alpha+a_2\alpha^2\mid
    a_i\in\K,\alpha=t+\left< f \right>, f(\alpha)=\alpha^3+\alpha+1=0 \right\}\]
    y como $\K$ tiene característica 2
    \[0=f(\alpha)^2=(\alpha^3+\alpha+1)^2=(\alpha^3)^2+\alpha^2+1=(\alpha^2)^3+\alpha^2+1=f(\alpha^2)\]
    y de forma análoga $0=f(\alpha)^4=f(\alpha^4)$. Por tanto deducimos que $f$
    se escinde sobre $\L$.
\end{ej}

\fecha{18/02}
\section{Elementos algebraicos y transcendentes}
\begin{defin}
    Sea $\K\subseteq\L$ una extensión de cuerpos y $\alpha\in\L$.
    \begin{itemize}
        \item $\alpha\in \L$ es \textit{algebraico} sobre $\K$ si existe
            $f\in\K[t]$ no nulo tal que $f(\alpha)=0$.
        \item $\alpha\in \L$ es \textit{transcendente} sobre $\K$ si $\alpha$ no
            es algebraico sobre $\K$.
    \end{itemize}
    Si $\alpha\in\C$ es algebraico sobre $\Q$ decimos simplemente que $\alpha$
    es un \textit{número algebraico}.
\end{defin}

\begin{ej}\hspace{0pt}
    \begin{itemize}
        \item $\sqrt{3}$ es algebraico sobre $\Q$ ya que es raíz de $t^2-3\in\Q[t]$
        \item $i$ es algebraico sobre $\R$ ya que es raíz de $t^2+1\in \R[t]$
            (también es algebraico sobre $\Q$).
    \end{itemize}
\end{ej}

\begin{tma}
    Los números $\pi$ y $e\in\R$ son trascendentes sobre $\Q$.
\end{tma}

\begin{ej}
    Sea $\K$ un cuerpo y $\L=\K(x)$ el cuerpo de fracciones del anillo de
    polinomios $\K[x]$. Entonces $x$ es trascendente sobre $\K$.
\end{ej}

\begin{proof}
    Supongamos que existe $f=a_0+a_1t+\ldots+a_nt^n\in\K[t]$ con $n\geq1$ tal
    que \[f(x)=a_0+a_1x+\ldots+a_nx^n=0\] en $\L=\K(x)$. Esto pasa si y solo sí el
    polinomio $f(x)\in\K[x]$ es el polinomio nulo: $a_0=a_1=\ldots=a_n=0$. Por
    tanto $x$ no es algebraico sobre $\K$.
\end{proof}

\begin{defin}
    Sea $\K\subseteq\L$ una extensión y $\alpha\in\L$ algebraico sobre $\K$. El
    \textit{polinomio mínimo} de $\alpha$ sobre $\K$ es el polinomio mónico de
    menor grado en $\K[t]$ que tiene a $\alpha$ como raíz. Lo denotaremos por
    $m_{\alpha,\K}$.
\end{defin}

\begin{ej}
    $x^2-3=m_{\sqrt{3},\Q}$ y $x^2+1=m_{i,\Q}$ ya que $\sqrt{3},i\not\in\Q$ y el
    polinomio mínimo no puede tener grado 1.
\end{ej}

\begin{prop}
    Sea $\K\subseteq\L$ una extensión y $\alpha\in \L$ algebraico sobre $\K$,
    \begin{enumerate}
        \item El polinomio mínimo $m_{\alpha,\K}$ es irreducible en $\K[t]$.
        \item Si $g\in \K[t]$ es tal que $g(\alpha)=0$ entonces $m_{\alpha,\K}$
            divide a $g$.
    \end{enumerate}
\end{prop}

\begin{proof}\hspace{0pt}
    \begin{enumerate}
        \item Sea $n=\deg{m_{\alpha,\K}}$. Supongamos que $m_{\alpha,\K}=f*h$ en
            $\K[t]$. Evaluando en $\alpha$:
            \[m_{\alpha,\K(\alpha)}=0=f(\alpha)g(\alpha)\text{ en }\L.\]
            Por tanto $f(\alpha)=0$ o $h(\alpha)=0$ al ser $\L$
            íntegro. Podemos suponer que $f(\alpha)=0$. Por definición de
            polinomio mínimo $\deg{f}\geq\deg{m_{\alpha,\K}}$ y como
            $f\mid m_{\alpha,\K}$ además $\deg{f}\leq \deg{m_{\alpha,\K}}$ de
            donde deducimos que $\deg{f}=\deg{m_{\alpha,\K}}$ y $\deg{h}=0$.
            Entonces $h\in\K\setminus\{0\}\implies m_{\alpha,\K}$ es
            irreducible.
        \item Sea $g\in\K[t]$ con $g(\alpha)=0$. Dividiendo $g$ por
            $m_{\alpha,\K}$ obtenemos $q,r\in\K[t]$ tales que \[g=q*m_{\alpha,
            \K}+r\] con $\deg{r}<\deg{m_{\alpha,\K}}$ o $r=0$
            ($\deg{0}\coloneqq-\infty$). Evaluando en $\alpha$:
            \[0=g(\alpha)=q(\alpha)*m_{\alpha,\K}+r(\alpha)=r(\alpha)\]
            y por tanto $r(\alpha)=0$. Si $r$ fuera no nulo se tendría que
            $r\in\K[x]$ y $r\neq 0$ con $\deg{r}<\deg{m_{\alpha,\K}}$ y
            $r(\alpha) = 0$, lo que contradice la definición de polinomio
            mínimo.
    \end{enumerate}
    \vspace{-1em}
\end{proof}


\begin{cor}
    Sea $\K\subseteq\L$ una extensión y $\alpha\in\L$ algebraico sobre $\K$. Si
    $f\in\K[t]$ es mónico e irreducible tal que $f(\alpha)=0$ entonces
    $f=m_{\alpha,\K}$.
\end{cor}

\begin{proof}
    Como $f(\alpha)=0$, por la proposición anterior existe $g\in\K[t]$ tal que
    $f=g*m_{\alpha,\K}$. Al ser $f$ irreducible uno de los dos factores está en
    $\K\setminus\{0\}$. No puede ser $m_{\alpha,\K}$ ya que tiene grado
    $n\geq1$, por lo que $g\in\K\setminus\{0\}$. Como $f$ y $m_{\alpha,\K}$ son
    mónicos el coeficiente de $t^n$ en ambos es $1$, luego $g=1$ y
    $f=m_{\alpha,\K}$.
\end{proof}

\begin{obs}
    Una consecuencia directa de este corolario es que el polinomio mínimo es
    único.
\end{obs}

\begin{ej}
    ¿Cual es el polinomio mínimo de $\sqrt{2}+\sqrt{3}$ sobre $\Q$?

    Consideramos el polinomio
    \begin{align*}
        f&=(t-\sqrt{2}-\sqrt{3})(t-\sqrt{2}+\sqrt{3})(t+\sqrt{2}-\sqrt{3})(t+\sqrt{2}+\sqrt{3})\\
         &=((t-\sqrt{2})^2-3) ((t+\sqrt{2})^2-3)\\
         &=(t^2-2\sqrt{2}t+2-3)(t^2+2\sqrt{2}t+2-3)\\
         &=(t^2-1-2\sqrt{2}t)(t^2-1+2\sqrt{2}t)\\
         &=(t^2-1)^2-8t^2=t^4-10t^2+1\in\Q[t]
    \end{align*}
    Como $\sqrt{2}+\sqrt{3}$ es raíz de $f$, si vemos que $f$ es irreducible en
    $\Q[t]$, tenemos que $f=m_{\sqrt{2}+\sqrt{3},\Q}$.
    \begin{itemize}
        \item $f$ no tiene raíces en $\Q$: Como $f\in\Z[t]$ es mónico sus
            posibles raíces racionales son enteros que dividen al término
            independiente. Como $\pm1$ no son raíces de $f$ deducimos que $f$ no
            tiene raíces racionales, es decir $\pm\sqrt{2}\pm\sqrt{3}\not\in\Q$.
            Esto implica que $f$ no tiene factores de grado $1$.
        \item Por el Lema de Gauss si $f$ fuera producto de dos polinomios de
            grado dos en $\Q[t]$ entonces existirían $g,h\in\Z[t]$ tales que
            $f=g*h$ con $\deg{g}=2$ y $\deg{h}=2$. Como $f$ es mónico podemos
            suponer que $g$ y $h$ son mónicos:
            \[f=t^4-10t^2+1=(t^2+at+b)(t^2+ct+d)=t^4+(a+c)t^3+(ac+b+d)t^2+(ad+bc)t+bd\]
            Igualando coeficientes obtenemos el sistema:
            \[
                \begin{cases}
                    a+c &= 0\\
                    ac+b+d &= -10\\
                    ad+bc&=0\\
                    bd&=1
                \end{cases}
                \implies
                \begin{cases}
                    a&=c\\
                    -c^2+b+d&=-10\\
                    +c(-d+b)&=0\\
                    bd&=1
                \end{cases}
            \]
            Como $b,d\in \Z$ tenemos dos casos:
            \begin{itemize}
                \item Caso 1: $b=d=1$. $-c^2+2=-10\Leftrightarrow c^2=12$ no
                    tiene solución en $\Z$.
                \item Caso 2: $b=d=-1$. $-c^2-2=-10\Leftrightarrow c^2=8$ no
                    tiene solución en $\Z$.
            \end{itemize}
            Por tanto, al no haber soluciones enteras, llegamos a un absurdo y
            $f$ es irreducible.
    \end{itemize}
\end{ej}

\fecha{22/02}
\section{Extensiones finitamente generadas}
Recordemos que si $\K$ es un cuerpo, $\K[x_1,\ldots,x_n]$ es el anillo de
polinomios en las variables $x_1,\ldots,x_n$ que se puede definir de forma
inductiva como \[\K[x_1,\ldots,x_n]=\K[x_1,\ldots,x_{n-1}][x_n]=\left\{
\sum_{\substack{I=\left( i_1,\ldots,i_n \right)\\ 0\leq i_j\leq N }}
a_Ix_1^{i_1}*\ldots*x_n^{i_n}\mid N\in \N,a_I\in\K\right\} \]

\begin{notacion}
    Si $\K\subseteq\L$ es una extensión y $\alpha_1,\ldots,\alpha_n\in\L$
    \[K[\alpha_1,\ldots,\alpha_n]=\left\{ p(\alpha_1,\ldots,\alpha_n)\mid p\in\K[x_1,
     \ldots,x_n]\right\} \subseteq \L\]
    \[K(\alpha_1,\ldots,\alpha_n)=\left\{\frac{\alpha}{\beta}=\alpha*\beta^{-1}
     \mid \alpha,\beta\in\K[\alpha_1,\ldots,\alpha_n]\text{ y }\beta\neq0
     \right\}\]
\end{notacion}

\begin{prop}
    $\K(\alpha_1,\ldots,\alpha_n)\subseteq\L$ es el menor subcuerpo de $\L$ que
    contiene a $\K$ y a $\alpha_1,\ldots,\alpha_n$.
\end{prop}

\begin{proof}
    Queda como ejercicio comprobar que $\K(\alpha_1,\ldots,\alpha_n)$ es un
    subcuerpo de $\L$. Supongamos que $\K\subseteq\F\subseteq\L$ con $\F$
    subcuerpo de $\L$ y $\alpha_1,\ldots,\alpha_n\in\F$. Si $p\in\K[x_1,\ldots,
    x_n]\subseteq\F$ se tiene que $p(\alpha_1,\ldots,\alpha_n)\in \F$ porque
    $\F$ es cerrado con respecto a la suma y multiplicación y los coeficientes
    de $p$ están en $\F$. Por tanto $\K[\alpha_1,\ldots,\alpha_n]\subseteq\F$.
    Como $\F$ es cuerpo vemos que $\K(\alpha_1,\ldots,\alpha_n)\subseteq\F$.
\end{proof}

\begin{defin}
    Una extensión $\K\subseteq\L$ es finitamente generada si existen
    $\alpha_1,\ldots,\alpha_n\in\L$ tales que $K(\alpha_1,\ldots,\alpha_n)=L$.
\end{defin}

\begin{ej}
    Sea
    $f=t^4-2=(t-\sqrt[4]{2})(t+\sqrt[4]{2})(t-i\sqrt[4]{2})(t+i\sqrt[4]{2})$,
    $\Q(\sqrt[4]{2},-\sqrt[4]{2},i\sqrt[4]{2},-i\sqrt[4]{2})=\Q(i,\sqrt[4]{2})$
\end{ej}

\begin{prop}
Si $\L=\K(\alpha_1,\ldots,\alpha_n)$ y $1\leq r\leq
n-1$,\[\L=\K(\alpha_1,\ldots,\alpha_r)(\alpha_{r+1},\ldots,\alpha_n)\]
\end{prop}

\begin{proof}
    Ejercicio. Usar que $\K(\alpha_1,\ldots,\alpha_r)$ es el menor subcuerpo de
    $\L$ que contiene a $\K$ y a $\alpha_1,\ldots,\alpha_r$.
\end{proof}

Sean $\K\subseteq\L$ una extensión de cuerpos y $\beta\in\L$ trascendente, vamos
a estudiar propiedades de la extensión $\K\subseteq\K(\beta)$

\begin{prop}
    Si $\beta\in\L$ es trascendente sobre $\K$ el homomorfismo de evaluación en
    $\beta$ \[\phi:\K[t]\to L \text{ tal que } p(t)\mapsto p(\beta)\] verifica
    que
    \begin{enumerate}
        \item $\ker{\phi}=\{0\}$
        \item $\im{\phi}=\K[\beta]$
        \item Se tiene un isomorfismo de $\K(t)\to \K(\beta)$ tal que
            $\frac{p(t)}{q(t)}\mapsto \frac{p(\beta)}{q(\beta)}$
    \end{enumerate}
\end{prop}

\begin{proof}\hspace{0pt}
    \begin{enumerate}
        \item Si $f=a_0+a_1t+\ldots+a_nt^n\in\K[t]$ es no nulo
            $\phi(f(t))=f(\beta)\neq0$ ya que $\beta$ es trascendente. Por tanto
            $\ker{\phi}=\{0\}$.
        \item $\K[\beta]=\left\{ p(\beta)\mid
            p\in\K[t]\right\}=\phi(\K[t])=\im{\phi}$
        \item Por el apartado anterior tenemos un isomorfismo
            $\K[t]\to \K[\beta]$  tal que $p(t)\mapsto p(\beta)$. Este
            isomorfismo se extiende a un isomorfismo de su cuerpo de fracciones.
    \end{enumerate}
\end{proof}

Veamos ahora el caso en el que $\beta$ es algebraico.

\begin{lema}
    Si $\beta\in\L$ es algebraico sobre $\K$, sea $\phi:\K[t]\to L$ el
    homomorfismo de evaluación en $\beta$. Entonces
    \begin{enumerate}
        \item $\im{\phi}=\K[\beta]$
        \item $\ker{\phi}=\left<m_{\beta,\K}\right>$
        \item Se tiene un isomorfismo \[\bar{\phi}:\frac{\K[t]}{\left<
            m_{\beta,\K} \right> }\to \K[\beta] \text{ tal que } p(t)+\left<
        m_{\beta,\K} \right> \mapsto p(\beta).\]
        En particular $\K[\beta]$ es un cuerpo.
    \end{enumerate}
\end{lema}

\begin{proof}\hspace{0pt}
    \begin{enumerate}
        \item Como en la proposición anterior.
        \item $m_{\beta,\K}\in\K[t]$ y $\phi(m_{\beta,\K})=m_{\beta,\K}(\beta)
            =0$ por lo que $m_{\beta,\K}\in \ker{\phi} \implies \left<
            m_{\beta,\K} \right> \subseteq \ker{\phi}$. Si $f\in\ker{\phi}
            \Leftrightarrow f(\beta)=0 \implies m_{\beta,\K}\mid f \implies f\in
            \left< m_{\beta,\K}\right> $, luego $\ker{\phi}\subseteq\left<
            m_{\beta,\K} \right> $
        \item Aplicamos el primer Teorema de isomorfía. Observamos que
            $\bar{\phi}(t+\left< m_{\beta,\K} \right>)=\beta$ y
            $\bar{\phi}(a+\left< m_{\beta,\K} \right>)=a$ para $a\in\K$. Como
            $m_{\beta,\K}$ es irreducible $\frac{\K[t]}{\left< m_{\beta,\K}
            \right> }$ es cuerpo luego $\K[\beta]$ es cuerpo.
    \end{enumerate}
\end{proof}

\begin{cor}
    Sea $\K\subseteq\L$ una extensión y $\beta\in \L$. Son equivalentes
    \begin{itemize}
        \item $\beta$ es algebraico sobre $\K$
        \item $\K[\beta]$ es cuerpo
    \end{itemize}
\end{cor}

\fecha{23/02}
\begin{proof}\hspace{1em}
    \begin{itemize}
        \item[$\implies$]  Si $\alpha$ es algebraico sobre $\K$, por la proposición
            anterior, sabemos que $\K[\alpha]\subseteq\L$ es un cuerpo. Por
            tanto $\K(\alpha)\subseteq\K[\alpha]$ ya que $\K(\alpha)$ es el
            menor subcuerpo de $\L$ que contiene a $\K$ y a $\alpha$. Por tanto
            $\K(\alpha)=\K[\alpha]$.
        \item[$\impliedby$]  Si $\alpha$ es trascendente $\K[\alpha]$ es
            isomorfo al anillo de polinomios $\K[x]$ que no es un cuerpo. Por
            tanto $\K[\alpha]\subseteq\K(\alpha)$.
        \item[$\impliedby$] Veamos otra forma de demostrar esta implicación. Si
            $\K[\alpha]=\K(\alpha)$, entonces $\K[\alpha]$ es un cuerpo. Puedo
            suponer que $\alpha\neq0$, por lo que existe $\alpha^{-1}\in
            \K[\alpha]$. Por tanto existen $a_i\in \K$ tales que
            \[\alpha^{-1}=a_0+a_1\alpha+\ldots+a_n+\alpha^n\] y multiplicando
            por $\alpha$ \[0=-1+a_0\alpha+a_1\alpha^2+\ldots+a_n\alpha^{n+1}.\]
            Por tanto $\alpha$ es raíz de un polinomio no nulo en $\K[t]$
            $\implies$
            $\alpha$ es algebraico sobre $\K[t]$.
    \end{itemize}
\end{proof}

\begin{cor}
    Si $\K\subseteq\L$ es una extensión y $\alpha_1,\ldots,\alpha_n\in\L$ son
    algebraicos sobre $\K$ entonces \[\K[\alpha_1,\ldots,\alpha_n]=
    \K(\alpha_1,\ldots,\alpha_n)\]
\end{cor}

\begin{proof}\hspace{0pt}
    \begin{itemize}
        \item Primero probamos que $\K[\alpha_1,\ldots,\alpha_n]$ es un cuerpo, por
    inducción sobre $n$.
    \item $\K(\alpha_1,\ldots,\alpha_n)=\left\{\text{menor subcuerpo que contiene a } \K \text{ y a } \alpha_1,\ldots,\alpha_n\right\}\subseteq\K[\alpha_1,\ldots,\alpha_n]\subseteq
    \K(\alpha_1,\ldots,\alpha_n)$
    \begin{itemize}
        \item $n=1$. Ya visto en el corolario anterior.
        \item $n>1$. Por hipótesis de inducción
            $\K[\alpha_1,\ldots,\alpha_{n-1}]$ es un cuerpo al ser
            $\alpha_1,\ldots,\alpha_{n-1}$ algebraicos sobre $\K$. Ahora, como
            $\alpha_n$ es algebraico sobre $\K$, es raíz del polinomio mínimo
            $m_{\alpha,\K}\in \K[t]$. Consideramos la extensión de cuerpos
            $\K\subseteq\K[\alpha_1,\ldots,\alpha_n]$. $\alpha_n$ es raíz del
            polinomio mínimo $m_{\alpha,\K}\in
            \K[\alpha_1,\ldots,\alpha_{n-1}][t]$, es decir $\alpha$ es
            algebraico sobre $\K[\alpha_1,\ldots,\alpha_{n-1}]$ y por tanto
            \begin{align*}
                \K[\alpha_1,\ldots,\alpha_{n}]
                &=\K[\alpha_1,\ldots,\alpha_{n-1}][\alpha_n]
                =\K[\alpha_1,\ldots,\alpha_{n-1}](\alpha_n)\\
                &=\K(\alpha_1,\ldots,\alpha_{n-1})(\alpha_n)
                =\K(\alpha_1,\ldots,\alpha_n).
            \end{align*}
    \end{itemize}\vspace{-1em}
    \end{itemize}
\end{proof}

\begin{ej}
    \[\Q(\sqrt{2},\sqrt{3})=\Q[\sqrt{2},\sqrt{3}]=\left\{ \sum_{0\leq
    i,j\leq n} a_j(\sqrt{2})^i(\sqrt{3})^j\mid a_{ij}\in\Q,n\in\N\right\}\]
    \[(\sqrt{2})^i(\sqrt{3})^i=\begin{cases}
        2^k*3^l &\text{si } i=2k, j=2l\\
        2^k*3^l*\sqrt{2}&\text{si } i=2k+1,j=l\\
        2^k*3^l*\sqrt{3}&\text{si } i=2k,j=2l+1\\
        2^k*3^l*\sqrt{6}&\text{si } i=2k+1,j=2l+1\\
    \end{cases}\]
    Y por tanto $\Q[\sqrt{2},\sqrt{3}]=\left\{
    a*1+b\sqrt{2}+c\sqrt{3}+d\sqrt{6}\mid a,b,c,d\in \Q    \right\} $
\end{ej}

Si $\K\subseteq\L$ es una extensión podemos ver $\L$ como espacio vectorial
sobre $\K$ con las operaciones $+$ de $\L$ y el producto de elementos de $\K$
por elementos de $\L$.

\begin{defin}
    El grado de la extensión $\K\subseteq\L$ es la dimensión de $\L$ visto como
    espacio vectorial sobre $\K$. Lo denotamos por $\left[\L:\K\right]$.
\end{defin}

\begin{ej}
    $[\C:\R]=2$ ya que todo número complejo se escribe de forma única como
    $a*1+b*i$ con $a,b\in \R$. Es decir $\left\{ 1,i \right\}$ es una base de
    $\C$ como espacio vectorial sobre $\R$.
\end{ej}

\begin{obs}
    $[\L:\K]=1 \Leftrightarrow\K=\L$
    \completar
    % ¿hacer dibujito?
\end{obs}

\begin{prop}
    Sea $\K\subseteq\L$ una extensión y $\alpha\in\L$ algebraico sobre $\K$.
    Entonces
    \begin{enumerate}
        \item $[\K[\alpha]:\K]=\deg{m_{\alpha,\K}}=n$
        \item $\left\{ 1,\alpha,\ldots,\alpha^{n-1} \right\} $ es base de $\K[\alpha]$ como espacio vectorial sobre $\K$.
    \end{enumerate}
\end{prop}

\begin{proof}
    Sabemos que $\K[\alpha]\cong \frac{\K[t]}{\left< m_{\alpha,\K} \right> }$ y
    \[\K[\alpha]=\left\{a_0+a_1\alpha+\ldots+a{n_a}\alpha^{n-1}\mid a_i\in\K\right\}\]
    por tanto $1,\alpha,\ldots,\alpha^{n-1}$ son generadores de $\K[\alpha]/\K$.
    Si fueran linealmente dependientes, existirían $a_0,\ldots,a_{n-1}\in\K$ no
    todos nulos tales que $a_01+a_1\alpha+\ldots+a_{n-1}\alpha^{n-1}=0$ lo que
    equivale a que exista $f=a_0+a_1t+\ldots+a_{n-1}t^{n-1}\in \K[t]$ con
    $f(\alpha)=0$ y $\deg{f}<\deg{m_{\alpha,\K}}=n$, que es una contradicción.
\end{proof}

\fecha{24/02}
\begin{ej}\hspace{0pt}
    \begin{itemize}
        \item Consideramos la extensión $\Q\subseteq\Q(\sqrt{2})$. El grado de
            la extensión es $[\Q(\sqrt{2}):\Q]=2$ ya que $m_{\sqrt{2},\Q}=t^2
            -2$. Por tanto $1,\sqrt{2}$ son base de $\Q(\sqrt{2})$ sobre $\Q$.
        \item Sea $\beta=\sqrt{2}+\sqrt{3}$, $m_{\beta,\Q}=t^4-10t^2+1$,
            $[\Q(\beta):\Q]=4$ y $1,\beta,\beta^2,\beta^3$ es una base de
            $\Q(\beta)$ sobre $\Q$, por lo que $\Q(\beta)=\left\{
            a+b\beta+c\beta^2+d\beta^3\mid a,b,c.d\in \Q \right\} $
        \item $\pi$ es trascendente sobre $\Q$, por lo que
            $1,\pi,\pi^2,\ldots,\pi^n,\ldots$ son linealmente independientes
            sobre $\Q$.
    \end{itemize}
\end{ej}

Recordemos que $\K\subseteq\L$ es una extensión finita si $[L:K]\in\N$.

\begin{tma}[transitividad del grado]
    Si tenemos extensiones $\F\subseteq\K\subseteq\L$ entonces:
    \begin{enumerate}
        \item $\F\subseteq\K$ y $\K\subseteq\L$ son finitas si y solo si
            $\F\subseteq\L$ es finita. En tal caso además
            \[[\L:\K][\K:\F]=[\L:\F].\]
        \item Si $[\K:\F]=\infty$ o $[\L:\K]=\infty$ entonces $[\L:\F]=\infty$.
    \end{enumerate}
\end{tma}

\begin{proof}\hspace{0pt}
    \begin{enumerate}
        \item[(2)] Por contrarrecíproco. Es equivalente probar que si
            $[\L:\F]\in \N\implies[\K:\F]\in \N \wedge[\L:\K]\in \N$. Sea
            $n=[\L:\F]$,
            \begin{itemize}
                \item Como $\K\subseteq\L$ es un subcuerpo, $\K$ es un
                    subespacio de $\L$, por lo que $\dim_\F{\K}=[\K:\F]\leq
                    \dim_\F{L}=n$
                \item Sea $\left\{ \gamma_1,\ldots,\gamma_n \right\}$ una base
                    de $\L$ sobre $\F$ y $\gamma\in \L$. Existen $a_i\in\F$ tales
                    que $\gamma=a_1\gamma_1+\ldots+a_n\gamma_n$, por lo que
                    $\left\{ \gamma_1,\ldots,\gamma_n \right\} $ son un sistema
                    de generadores de $\L$ sobre $\K$ y entonces
                    $\dim_\K{\L}<n$.
            \end{itemize}
        \item[(1)] Sean $m=[\L:\K]$ y $s=[\K:\F]$ queremos ver que
            $[\L:\F]=m*s$.

            Sean $\alpha_1,\ldots,\alpha_m$ base de $\L$ sobre
            $\K$ y $\beta_1,\ldots,\beta_s$ base de $\K$ sobre $\F$ veamos que
            \[B=\left\{\beta_j*\alpha_i\mid j=1,\ldots,s,\ i=1,\ldots,m\right\}\]
            es una base de $\L$ sobre $\F$. Sea $\gamma\in\L$ existen $a_i\in\K$
            tales que $\gamma=a_1\alpha_1+\ldots+a_n\alpha_m$ y para cada $a_i$
            existen $b_{ij}\in\F$ tales que $a_i=b_{i1}\beta+\ldots+b_{is}
            \beta_s$. Con esto se tiene que \[\gamma=\sum_{i=1}^{m}a_i\alpha_i=
            \sum_{i=1}^m\sum_{j=1}^sb_{ij}\beta_j\alpha_i\] y por tanto $B$ es
            un sistema de generadores de $\L$ sobre $\F$.

            Veamos que son linealmente independientes. Sean $d_{ij}\in\F$ tales
            que
            \[0=\sum_{i=1}^m(\underbrace{\sum_{j=1}^sd_{ij}\beta_j}_{=e_i\in\K})
            \alpha_i=\sum_{i=1}^me_i\alpha_i\implies\forall i: 0=e_i=\sum_{j=1}^sd_{ij}
            \beta_j\implies\forall i,j: d_{ij}\]
            ya que $\alpha_1,\ldots,\alpha_m$ son L.I. sobre $\K$ y
            $\beta_1,\ldots,\beta_s$ son L.I. sobre $\F$.
    \end{enumerate}
\end{proof}

\begin{ej}
    El polinomio mínimo de $\sqrt{2}$ sobre $\Q(\sqrt{3})$ es
    $t^2-2$. Esto es equivalente a que
    $\sqrt{2}\not\in\Q(\sqrt{3})=\left\{ a+b\sqrt{3}\mid a,b\in\Q\right\}$. Si
    $\sqrt{2}\in\Q(\sqrt{3)}$ existirían $a,b\in\Q$ tales que
    $\sqrt{2}=a+b\sqrt{3}$. Pero \[2=(a+b\sqrt{3})^2=a^2+3b^2+2ab\sqrt{3}
        \Leftrightarrow (a^2+3b^2-2)+2ab\sqrt{3}=0 \implies 2ab=0 \wedge
        a^2+3b^2-2=0\]
     Si $b=0$, $a=\sqrt{2}$, lo que no puede ser ya que $\sqrt{2}\not\in\Q$ y
     si $a=0$, $b=\sqrt{2}=b\sqrt{3} \Leftrightarrow b={\sqrt{2}}/{\sqrt{3}}=
     \sqrt{{2}/{3}}$, lo que tampoco es posible porque $\sqrt{{2}/{3}}\not\in\Q$.

    Sean ahora las extensiones $\Q\subseteq\Q(\sqrt{3})\subseteq
    \Q(\sqrt{3},\sqrt{2})$, por el teorema anterior sabemos que
    \[[\Q(\sqrt{3},\sqrt{2}):\Q]=\underbrace{[\Q(\sqrt{3}):\Q]}_{=2}\underbrace{
      [\Q(\sqrt{3},\sqrt{2})=\Q(\sqrt{3})(\sqrt{2}):\Q(\sqrt{3})]}_{=2}=4.\]
    Como $m_{\sqrt{3},\Q}=t^2-3$, $1,\sqrt{3}$ es base de $\Q(\sqrt{3})$ sobre
    $\Q$ y como $m_{\sqrt{2},\Q(\sqrt{3})}=t^2-2$ (ejercicio), $1,\sqrt{2}$ es
    base de $\Q(\sqrt{2})$ sobre $\Q$. Por la demostración del teorema anterior
    sabemos que $\left\{1,\sqrt{2},\sqrt{3},\sqrt{3}\sqrt{2}=\sqrt{6}\right\}$
    es base de $\Q(\sqrt{2},\sqrt{3})$ sobre $\Q$.
\end{ej}

\begin{ejer}\hspace{0pt}
    \begin{enumerate}
        \item Probar que $\Q(\sqrt{2}+\sqrt{3})=\Q(\sqrt{2},\sqrt{3})$
        \item Hallar el polinomio mínimo $m_{\alpha,\Q}$, siendo:
            \begin{multicols}{4}
                \begin{enumerate}
                    \item $\alpha=\sqrt{2}(1+i)$
                    \item $\alpha=\sqrt{2+\sqrt{2}}$
                    \item $\alpha=\sqrt[n]{2}$
                \end{enumerate}
            \end{multicols}
        \item Hallar los grados de las extensiones
            \begin{multicols}{2}
                \begin{enumerate}
                    \item $\Q\subseteq\Q(i,\sqrt[4]{2})$
                    \item $\Q\subseteq\Q\left(\sqrt{3},\sqrt[3]{2}\right)$
                    \item $\Q\subseteq\Q\left(\sqrt{2+\sqrt{2}}\right)$
                    \item $\Q\subseteq\Q\left(i,\sqrt{2+\sqrt{2}}\right)$
                \end{enumerate}
            \end{multicols}
    \end{enumerate}
\end{ejer}

\fecha{25/02}
\begin{sol}
    \begin{enumerate}
        \item La inclusión $\subseteq$ es inmediata ya que $\sqrt{2}+\sqrt{3}\in
            \Q(\sqrt{2},\sqrt{3})$. Tenemos entonces la cadena
            \[Q\subseteq\Q(\sqrt{2}+\sqrt{3})\subseteq\Q(\sqrt{2},\sqrt{3}).\]
            La extensión $\Q\subseteq\Q(\sqrt{2},\sqrt{3})$ tiene grado 4 y si
            consideramos $\Q\subseteq\Q(\sqrt{2}+\sqrt{3})$ el polinomio mínimo
            $m_{\sqrt{2}+\sqrt{3},\Q}=t^4-10t^2+1$ tiene grado 4, por lo que la
            extensión también. Usando la transitividad del grado:
            \[\underbrace{[\Q(\sqrt{2},\sqrt{3}):\Q]}_{4}=
                \underbrace{[\Q(\sqrt{2},\sqrt{3}):\Q(\sqrt{2}+\sqrt{3})]}_{1}*
                \underbrace{[\Q(\sqrt{2}+\sqrt{3}):\Q]}_{4}\]
                Y como hemos visto, esto
                $[\Q(\sqrt{2},\sqrt{3}):\Q(\sqrt{2}+\sqrt{3})]=1\Leftrightarrow
                \Q(\sqrt{2},\sqrt{3})=\Q(\sqrt{2}+\sqrt{3})$.
        \item
            \begin{enumerate}
                \item El polinomio $f=t^4+16$ verifica $f(\alpha)=
                    f(\sqrt{2}(1+i))=0$:

                    Sea $f=(t+\sqrt{2}+i\sqrt{2})(t+\sqrt{2}-i\sqrt{2})
                    (t-\sqrt{2}+i\sqrt{2})(t-\sqrt{2}-i\sqrt{2})=\ldots=t^4+16$.

                    De otra forma, como $\alpha=\sqrt{2}(1+i)$,
                    $\alpha^2=2(1+i)^2=4i$ y $\alpha^4=-16$, luego
                    $f(\alpha)=\alpha^4+16=0$.

                    Ahora falta ver que $f$ es irreducible:
                    Como $f=t^4+16=t^4+2^4$ y $f$ es irreducible en $\K[t]
                    \Leftrightarrow g=f(at), a\in \K\setminus\{0\}$ es
                    irreducible, veamos que $g=f(2t)=2^4t^4+2^4=2^4(t^4+1)$ es
                    irreducible. Por el criterio de traslación, $g$ es
                    irreducible $\Leftrightarrow h=g(t+a), a\in\K$ es
                    irreducible. Sea $h=g(t+1)=(t+1)^4+1=t^4+4t^3+6t^2+4t+2$,
                    por el criterio de Eisenstein con $p=2$, $h$ es irreducible
                    en $\Q[t]$.
                \item Como antes, puedo obtener el polinomio de dos formas:
                    \begin{align*}
                        f&=(t-\sqrt{2+\sqrt{2}})(t+\sqrt{2+\sqrt{2}})
                        (t-\sqrt{2-\sqrt{2}})(t+\sqrt{2-\sqrt{2}})\\
                         &=(t^2-(2+\sqrt{2}))(t^2-(2-\sqrt{2}))=(t^2-2)^2-2\\
                         &=t^4-4t^2+2
                    \end{align*}
                    O también manipulando la expresión de $\alpha$:
                    \[\alpha=\sqrt{2+\sqrt{2}}\implies\alpha^2=2+\sqrt{2}
                     \implies\alpha^2-2=\sqrt{2}\implies(\alpha^2-2)^2=2\implies
                     \alpha^4-4\alpha^2+2=0\]

                     Usando el criterio de Eisenstein con $p=2$ se ve que $f$ es
                     irreducible.
                \item Procediendo como en los apartados anteriores se obtiene
                    $f(t)=t^n-2$, que es irreducible por el criterio de
                    Eisenstein con $p=2$.
            \end{enumerate}
        \item
            \begin{enumerate}
                \item Si consideramos las extensiones intermedias
                    $\Q\subseteq\Q(\sqrt[4]{2})\subseteq\Q(\sqrt[4]{2},i)$,
                    calculando su grado, podemos aplicar la transitividad del
                    grado. Como $m_{\sqrt{2},\Q}=t^4-2$,
                    $[\Q(\sqrt[4]{2}):\Q]=4$.
            \end{enumerate}
    \end{enumerate}
\end{sol}
\fecha{1/03}
\section{Extensiones algebraicas}
\begin{defin}
    Una extensión $\K\subseteq\L$ es algebraica si $\forall\alpha\in\L$ $\alpha$
    es algebraico sobre $\K$.
\end{defin}

\begin{prop}
    Sea $\K\subseteq\L$ una extensión finita
    \begin{enumerate}
        \item Si $\alpha\in \L$ entonces $\alpha$ es algebraico sobre $\K$ y
            $\deg{m_{\alpha,\K}}$ divide a $[\L]:\K$.
        \item $\K\subseteq\L$ es algebraica.
    \end{enumerate}
\end{prop}

\begin{proof}\hspace{0pt}
    \begin{enumerate}
        \item Si $\alpha\in\L$,  $\K\subseteq\K(\alpha)\subseteq\L$  y por la
            transitividad del grado
            \[\N\ni[\L:\K]=[L:\K(\alpha)]*\underbrace{[\K(\alpha):\K]}_{\deg{m_{\alpha,\K}}}.\]
        \item Todo $\alpha\in\L$ es algebraico sobre $\K$ por (1).
    \end{enumerate}
\end{proof}

\begin{tma}
    Sea $\K\subseteq\L$ una extensión. Son equivalentes
    \begin{enumerate}
        \item $\K\subseteq\L$ es finita ($[\L:\K]\in \N$).
        \item Existen $\alpha_1,\ldots,\alpha_n\in\L$ tales que son algebraicos
            sobre $\K$ y $\L=\K(\alpha_1,\ldots,\alpha_n)$
    \end{enumerate}
\end{tma}

\begin{proof}\hspace{0pt}
    \begin{itemize}
        \item (1)$\implies$(2). $[\L:\K]=\dim_\K{\L}=m\in\N$. Sea
            $\alpha_1,\ldots,\alpha_m$ una base de $\L$ sobre $\K$. Si
            $\gamma\in\L$ existen $a_1,\ldots,a_m\in \K$ tales que
            $\gamma=a_1\alpha_1+\ldots+a_m\alpha_m \implies \L\subseteq
            \K(\alpha_1,\ldots,\alpha_m)\subseteq L$
        \item (2)$\implies$(1). Lo probamos por inducción sobre $n$.
            \begin{itemize}
                \item Si $n=1$ entonces $\L=\K(\alpha_1)$ con $\alpha_1$
                    algebraico sobre $\K \implies
                    [\L:\K]=\deg{m_{\alpha_1,\K}}\in\N$
                \item Sea $n>1$ y supongamos cierto el resultado para $n-1$. Sea
                    $\K'\coloneqq\K(\alpha_1,\ldots,\alpha_{n-1})$, $\L=\K'(\alpha_n)$.
                    Como $\alpha_n$ es algebraico sobre $K$ se tiene que
                    $m_{\alpha_n,\K}\in \K[t]\subseteq\K'[t]$ ya que
                    $\K\subseteq\K'$. Por tanto, como $m_{\alpha_n,\K}(\alpha)=
                    0$, $\alpha_n$ es algebraico sobre $\K'$. Entonces $[\L:\K']
                    \in\N$ (como en el caso $n=1$) y por hipótesis de inducción
                    $[\K':\K]\in \N$. Por la transitividad del grado
                    $[\L:\K]\in\N$.
            \end{itemize}
    \end{itemize}
\end{proof}

\begin{ejer}
    Sean $\L_0\subseteq L_1\subseteq\ldots\subseteq\L_n$ extensiones. Probar
    \begin{enumerate}
        \item Si $\forall i\in \left\{ 1,\ldots,n\right\}[\L_i:\L_{i-1}]\in\N$
            entonces $ [\L_n:\L_0]=[\L_n:\L_{n-1}]*\ldots*[\L_1:\L_0]$.
        \item Si existe $i\in \left\{ 1,\ldots,n\right\}$ tal que
            $[\L_i:\L_{i-1}]=\infty$ entonces $[\L_n:\L_0]$.
    \end{enumerate}
\end{ejer}

\begin{prop}
    Sea $\K\subseteq\L$ una extensióny $\alpha,\beta\in\L$. Si $\alpha,\beta$
    son algebraicos sobre $\K$ entonces $\alpha+\beta$ y $\alpha*\beta$ son
    algebraicos sobre $\K$.
\end{prop}

\begin{proof}
    $\K\subseteq\K(\alpha,\beta)\subseteq\L$. Como $\alpha,\beta$ son
    algebraicos sobre $\K$, $\K\subseteq\K(\alpha,\beta)$ es finita. Esto
    implica que $\alpha+\beta\in\K(\alpha,\beta)$ y $\alpha*\beta\in
    \K(\alpha,\beta)$ han de ser algebraicos sobre $\K$ ya que
    \[\K\subseteq\K(\alpha+\beta)\subseteq\K(\alpha,\beta)\]
    y entonces $\K\ni[\K(\alpha,\beta):\K]=[\K(\alpha,\beta):\K(\alpha+\beta)]*
    \underbrace{[\K(\alpha+\beta):\K]}_{\in \N}$ (análogamente para
    $\alpha*\beta$)).
\end{proof}

\begin{obs}
    Si $\K\subseteq\L$ es finita y $\alpha\in\L$ entonces $[\K(\alpha):\K]\in
    \N$ divide a $[\L:\K]\in\N$.
\end{obs}

\begin{ejer}
    Probar que si $\alpha\neq 0$ es algebraico sobre $\K$ entonces $\alpha^{-1}$
    es algebraico sobre $\K$.
\end{ejer}

Hay dos formas de demostrar esto:
\begin{enumerate}
    \item Hallar un polinomio no nulo que tenga a $\alpha^{-1}$ como raíz.
    \item Observar que $\alpha^{-1}\in \K(\alpha),[\K(\alpha):\K]\in\N\implies
        [\K(\alpha^{-1}:\K]\in\N$.
\end{enumerate}

\begin{tma}
    Sea $\K\subseteq\L$ una extensión y \[M=\left\{ \alpha\in\L\mid\alpha\text{ es
    algebraico sobre }\K \right\}.\]
    Entonces $M$ es cuerpo y $\K\subseteq M\subseteq\L$.
\end{tma}

\begin{proof}
    Por la proposición anterior, si $\alpha,\beta\in
    M\implies\alpha+\beta,\alpha*\beta\in M$. Por tanto $M$ es subanillo de
    $\L$. Ahora $\K\subseteq M$ ya que si $\alpha\in\K$, $t-\alpha\in\K[t]$
    tiene a $\alpha$ como raíz. Por el ejercicio anterior, si $\alpha\in
    M\setminus\{0\}, \alpha^{-1}\in M$, de donde $M$ es un cuerpo.
\end{proof}

\begin{ej}
    Consideramos la extensión $\Q\subseteq\C$ y sea \[\overline{\Q}=\left\{ z\in
    \C\mid z\text{ es algebraico sobre }\Q\right\}=\left\{ \text{clausura
    algebraica de }\Q\right\}.\]

    $\overline{\Q}$ es un cuerpo y es algebraicamente cerrado.
\end{ej}

\fecha{2/03}
\begin{prop}\hspace{0pt}
    \begin{enumerate}
        \item Si $\F\subseteq\K$ y $\K\subseteq\L$ son extensiones algebraicas entonces
    $\F\subseteq\L$ es algebraica.
        \item Sean $\K\subseteq\L$ una extensión, $\F\subseteq\K$ una extensión
            algebraica y $\gamma\in\L$. Entonces $\gamma$ es algebraico sobre
            $\F$.
    \end{enumerate}
\end{prop}

\begin{proof}
        Observamos que basta con probar $(2)$.

        Como $\gamma$ es algebraico sobre $\K$, $\gamma$ es raíz de
        $m_{\gamma,\K}=t^n+a_{n-1}t^{n-1}+\ldots+a_1t+a_0\in\K[t]$. Por
        hipótesis $a_0,\ldots,a_{n-1}\in \K$ son algebraicos sobre $\F$.
        Consideramos las extensiones intermedias
        \[\F\subseteq\K':=\F(a_0,\ldots,a_{n-1})\subseteq\K.\]
        Entonces $m_{\gamma,\K}\in \K'[t]$ y $\gamma$  es
        algebraico sobre $\K' \implies \K'(\gamma)\supseteq \K'$ es una
        extensión finita. Tenemos que $\F\subseteq\K'=\F(a_0,\ldots,a_{n-1})$ es
        una extensión finita, por el teorema anterior (ya que
        $a_0,\ldots,a_{n-1})$ son algebraicos sobre $\F$). Por la transitividad
        del grado, como las extensiones
        \[\F\subseteq\F(a_0,\ldots,a_{n-1})=\K'\subseteq\F(a_0,\ldots,a_{n-1})(\gamma)=\K'(\gamma)\]
        son finitas, $\F\subseteq\K'(\gamma)$ es finita. Por tanto, como toda
        extensión finita es algebraica, $\gamma$ es algebraico sobre $\F$.
\end{proof}

Hemos visto que toda extensión finita es algebraica. Pero ¿toda extensión
algebraica es finita? Podemos considerar de nuevo el ejemplo de la clausura
algebraica de $\Q$.

\begin{ej}
    $[\Q(\sqrt[n]{2}):\Q]=n$ ya que $m_{\sqrt[n]{2} ,\Q}=t^n-2$. Por tanto si
    $\Q\subseteq\L\subseteq\C$ y suponemos que $\sqrt[n]{2}\in\L\forall n\geq2$
    entonces $\Q\subseteq\L$ no es finita. Si lo fuera y $[\L:\Q]=N\in\N$,
    \[\forall n\in \N: N=[\L:\Q(\sqrt[n]{2})]\underbrace{[\Q(\sqrt[n]{2}):
        \Q]}_{n}\implies n|N\] lo que es absurdo. En particular
        $\Q\subseteq\overline{\Q}$ y $\Q\subseteq\Q(\sqrt{2},\sqrt[3]{2},
        \sqrt[4]{2},\ldots)$ no son finitas.
\end{ej}

\begin{cor}
    El cuerpo $\overline{\Q}$ es algebraicamente cerrado.
\end{cor}

\begin{proof}
    Basta probar que si $f\in\overline{\Q}[t]$ con $\deg{f}\geq1$ entonces $f$
    tiene una raíz en $\overline{\Q}$. Como $\overline{\Q}\subseteq\C$ y por el
    teorema fundamental del álgebra existe $\alpha\in\C$ tal que $f(\alpha)=0
    \implies\alpha$ es algebraico sobre $\overline{\Q}$ y como $\overline{\Q}$
    es algebraico sobre $\Q$, $\alpha$ es algebraico sobre $\Q \implies
    \alpha\in\Q$.
\end{proof}

\begin{nota}
    Si $\K$ es un cuerpo entonces existe la clausura algebraica de $\K$. Se
    denota por $\overline{\K}$ y se tiene que la extensión $\K\subseteq
    \overline{\K}$ es única salvo isomorfismo. Se puede ver una demostración de
    este hecho en el libro de \textit{Fernando y Gamboa}.
\end{nota}

\section{Paréntesis: Grupos cíclicos y raíces de la unidad}

Si $\alpha\in\R$ recordemos que $e^{i\alpha}=\cos{\alpha}+i\sin{\alpha}$. Como
$e^{i\alpha}*e{i\alpha'}=e^{i(\alpha+\alpha')}$ la aplicación
$f:\R\to\C^\star=\C\setminus\{0\}$ tal que $f(\alpha)=e^{i\alpha}$ es un
homomorfismo de grupos entre $(\R,+)$ y $(\C^\star,*)$ con $\im{f}=\left\{ z\in
\C\mid \left| z \right| = 1 \right\}$.

Las raíces n-ésimas de $1$ son \[\eta_n=\left\{ z\in \C\mid z^n=1 \right\} =
\left\{ e^{2\pi ik/n}\mid k\in\left\{1,\ldots,n\right\}\right\}\left< \omega
\right> = \left\{ \omega,\ldots,\omega^{n-1},\omega^n=1 \right\}\]
donde $\omega = e^{2\pi i /n}$. Y por tanto forman un grupo cíclico.

\begin{obs}
    Si $G= \left< a \right>$ es un grupo cíclico de orden $n$
    \begin{enumerate}
        \item $\text{ord}(a^k)=\frac{n}{\text{mcd}(k,n)}$.
        \item $G= \left< a \right> \Leftrightarrow \text{mcd}(k,n)=1$.
        \item El número de generadores de $G$ es $\#\left\{ 1\leq k<
            n\mid\text{mcd}(k,n)=1 \right\}=\varphi(n)$.
    \end{enumerate}
\end{obs}

\end{document}
